\documentclass[10pt]{article}

\usepackage{fullpage}
\usepackage{setspace}
\usepackage{parskip}
\usepackage{titlesec}
\usepackage[section]{placeins}
\usepackage{xcolor}
\usepackage{breakcites}
\usepackage{lineno}
\usepackage{hyphenat}



\renewcommand{\familydefault}{\sfdefault}


\PassOptionsToPackage{hyphens}{url}
\usepackage[colorlinks = true,
            linkcolor = blue,
            urlcolor  = blue,
            citecolor = blue,
            anchorcolor = blue]{hyperref}
\usepackage{etoolbox}
\makeatletter
%\patchcmd\@combinedblfloats{\box\@outputbox}{\unvbox\@outputbox}{}{%
%  \errmessage{\noexpand\@combinedblfloats could not be patched}%
%}%
\makeatother


\usepackage{natbib}




\renewenvironment{abstract}
  {{\bfseries\noindent{\abstractname}\par\nobreak}\footnotesize}
  {\bigskip}

\titlespacing{\section}{0pt}{*3}{*1}
\titlespacing{\subsection}{0pt}{*2}{*0.5}
\titlespacing{\subsubsection}{0pt}{*1.5}{0pt}


\usepackage{authblk}


\usepackage{graphicx}
\usepackage[space]{grffile}
\usepackage{latexsym}
\usepackage{textcomp}
\usepackage{longtable}
\usepackage{tabulary}
\usepackage{booktabs,array,multirow}
\usepackage{amsfonts,amsmath,amssymb}
\providecommand\citet{\cite}
\providecommand\citep{\cite}
\providecommand\citealt{\cite}
% You can conditionalize code for latexml or normal latex using this.
\newif\iflatexml\latexmlfalse
\providecommand{\tightlist}{\setlength{\itemsep}{0pt}\setlength{\parskip}{0pt}}%

\AtBeginDocument{\DeclareGraphicsExtensions{.pdf,.PDF,.eps,.EPS,.png,.PNG,.tif,.TIF,.jpg,.JPG,.jpeg,.JPEG}}

\usepackage[utf8]{inputenc}
\usepackage[english]{babel}



\usepackage{float}



\usepackage[margin=1.5in]{geometry}






\begin{document}

\title{Wiley Interdisciplinary Reviews (WIREs)}



\author[1]{Alberto Pepe}%
\affil[1]{Authorea Team}%


\vspace{-1em}



  \date{}


\begingroup
\let\center\flushleft
\let\endcenter\endflushleft
\maketitle
\endgroup





\selectlanguage{english}
\begin{abstract}
The abstract should be a concise (less than 250 words) description of
the article and its implications. It should include all keywords
associated with your article, as keywords increase its discoverability.
Please do not include generic phrases such as ``This article discusses
\ldots{}'' or ``Here we review,'' or references to other articles. Note:
You will be required to copy this abstract into the submission system
when uploading your article.%
\end{abstract}%



\sloppy


\textbf{Remember that you are writing for an interdisciplinary audience.
Please be sure to discuss interdisciplinary themes, issues, debates,
etc. where appropriate.} Note that the WIREs are forums for review
articles, rather than primary literature describing the results of
original research.

\section*{Article Title}

{\label{350486}}

The title should not exceed 20 words. Please be original and try to
include keywords, especially before a colon if applicable, as they will
increase the discoverability of your
article.~\href{https://authorservices.wiley.com/author-resources/Journal-Authors/Prepare/writing-for-seo.html}{Tips
on Search Engine Optimization}

\section*{Article Type}

{\label{925764}}

The \href{http://wires.wiley.com/go/forauthors\#ArticleTypes}{Article
Type} denotes the intended level of readership for your article. Please
select one of the below article type options. An Editor may have
mentioned a specific Article Type in your invitation letter; if so,
please let them know if you think a different Article Type better suits
your topic.~

\begin{itemize}
\tightlist
\item
  Opinion
\item
  Primer
\item
  Overview
\item
  Advanced Review
\item
  Focus Article
\item
  Software Focus
\end{itemize}

\section*{Authors}

{\label{290010}}

List each person's full name, ORCID iD, affiliation, email address, and
any conflicts of interest. Please use an asterisk (*) to indicate the
corresponding author.

The preferred (but optional) format for author names is First Name,
Middle Initial, Last Name.~~

The submitting author is required to provide
an~\href{https://authorservices.wiley.com/author-resources/Journal-Authors/Submission/orcid.html}{ORCID
iD}, and all other authors are encouraged to do so.~

Wiley requires that all authors disclose any potential conflicts of
interest. Any interest or relationship, financial or otherwise, that
might be perceived as influencing an author's objectivity is considered
a potential conflict of interest. The existence of a conflict of
interest does not preclude publication.

\section*{Abstract}

{\label{468816}}

The abstract should be a concise (less than 250 words) description of
the article and its implications. It should include all keywords
associated with your article, as keywords increase its discoverability
(\href{https://authorservices.wiley.com/author-resources/Journal-Authors/Prepare/writing-for-seo.html}{Tips
on Search Engine Optimization}). Please try not to include generic
phrases such as ``This article discusses \ldots{}'' or ``Here we
review,'' or references to other articles. Note: You will be required to
copy this abstract into the submission system when uploading your
article.

Optional: If you would like to submit your abstract in an additional
language,~\href{http://wires.wiley.com/go/forauthors\#Resources}{read
more}.

\section*{Graphical/Visual Abstract and
Caption}

{\label{750712}}

Include an attractive full color image to go under the text abstract and
in the online Table of Contents.~\textbf{You will also need to upload
this as a separate file during submission.~}It may be a figure or panel
from the article or may be specifically designed as a visual summary.
While original images are preferred, if you need to look for a
thematically appropriate stock image, you can go
to~\href{http://pixabay.com/}{pixabay.com}~(not affiliated with Wiley)
to find a free stock image with a CC0 license. Another option you have
is to utilize professional illustrators with
Wiley's~\href{https://wileyeditingservices.com/en/article-preparation/graphical-abstract-design}{Graphical
Abstract Design service}.

Size: The minimum resolution is 300 dpi. Please keep the image as simple
as possible because it will be displayed in multiple sizes. Multiple
panels and text other than labels are strongly discouraged.

Caption: This is a narrative sentence to convey the article's essence
and wider implications to a non-specialist audience. The maximum length
is 50 words, but consider using 280 characters or less to facilitate
social media sharing, which can increase the discoverability of your
article.

\par\null

\section*{1. Introduction}

{\label{252565}}

Introduce your topic in \textasciitilde{}2 paragraphs,
\textasciitilde{}750 words.

While Wiley does consider articles on preprint servers (ArXiv, bioRxiv,
psyArXiv, SocArXiv, engrXiv, etc.) for submission to primary research
journals, preprint articles should not be cited in WIREs manuscripts as
review articles should discuss and draw conclusions only from
peer-reviewed research. Remember that original research/unpublished work
should also not be included as it has not yet been peer-reviewed and
could put the work in jeopardy of getting published in the primary
press.

Citations are automatically generated by Authorea. Select~\textbf{cite}
to find and cite bibliographic resources. The citations will
automatically be generated for you in APA format, the style used by most
WIREs titles. If you are writing for~\emph{WIREs Computational Molecular
Science} (WCMS), you will need to use the Vancouver reference and
citation style, so before exporting click Export-\textgreater{} Options
and select a Vancouver export style.

A sample citation:~~\hyperref[csl:1]{(Murphy et al., 2019)}

\section*{2. First Level Heading}

{\label{350277}}

Begin the main body of the text here, using a maximum of 3 levels of
headings (style as shown below and numbered). Try to create headings
that:

\begin{itemize}
\tightlist
\item
  help the reader find information quickly;~
\item
  are descriptive yet specific;
\item
  are compatible in phrasing and style; and~
\item
  are concise (less than 50 characters)
\end{itemize}

\subsection*{2.1 Second level heading}

{\label{733281}}

Text

\subsubsection*{2.1.1 Third level heading}

{\label{824595}}

Text

\par\null

\section*{Figures and Tables}

{\label{432859}}

Figures and tables should be numbered separately, in the order in which
they appear in the manuscript. Please embed them in the correct places
in the text to facilitate peer review. Permission to \textbf{reuse or
adapt} previously published materials must be submitted before article
acceptance.
\href{http://wires.wiley.com/go/forauthors\#Resources}{Resources}

Production quality figure files with captions are still to be submitted
separately.
\href{https://authorservices.wiley.com/asset/photos/electronic_artwork_guidelines.pdf}{Figure
preparation and formatting}

\textbf{Captions should stand alone} and be informative outside of the
context of the article. This will help educators who may want to use a
PowerPoint slide of your figure. Explain any abbreviations or symbols
that appear in the figure and make sure to include \textbf{credit lines}
for any previously published materials.

\section*{Conclusion}

{\label{880788}}

Sum up the key conclusions of your review, highlighting the most
promising scientific developments, directions for future research,
applications, etc. The conclusion should be \textasciitilde{}2
paragraphs, \textasciitilde{}750 words total.

\section*{Funding Information}

{\label{974317}}

You will be required to enter your funding information into the
submission system so that we can apply proper IDs to your funders and
help you comply with any funder mandates.

\section*{Research Resources}

{\label{808103}}

List sources of non-monetary support such as supercomputing time at a
recognized facility, special collections or specimens, or access to
equipment or services.
\href{https://orcid.org/organizations/research-orgs/resources}{Research
Resources} can also be added to ORCiD profiles. For biomedical
researchers, the \href{https://scicrunch.org/resources}{Resource
Identification Portal} supports NIH's new guidelines for Rigor and
Transparency in biomedical publications.

\section*{Acknowledgments}

{\label{749861}}

List contributions from individuals who do not meet the criteria for
authorship (for example, to recognize people who provided technical
help, collation of data, writing assistance, acquisition of funding, or
a department chairperson who provided general support), \textbf{with
permission} from the individual. Thanks to anonymous reviewers are not
appropriate.

\section*{Notes}

{\label{390481}}

Authors writing from a humanities or social sciences perspective may use
notes if a \textbf{comment or additional information} is needed to
expand on a citation. (Notes only containing citations should be
converted to references. Conversely, any references containing comments,
such as ``For an excellent summary of\ldots{},'' should be converted to
notes.) Notes should be indicated by \textbf{superscript letters}, both
in the text and in the notes list. Citations within notes should be
included in the reference section, as indicated below.

\section*{Further Reading}

{\label{153582}}

For readers who may want more information on concepts in your article,
provide full references and/or links to additional recommended resources
(books, articles, websites, videos, datasets, etc.) that are not
included in the reference section. Please do not include links to
non-academic sites, such as Wikipedia, or to impermanent websites.

\section*{\texorpdfstring{{Note About
References}}{Note About References}}

{\label{514168}}

References are automatically generated by Authorea.
Select~\textbf{cite~}to find and cite bibliographic resources. The
bibliography will automatically be generated for you in APA format, the
style used by most WIREs titles. If you are writing for~\emph{WIREs
Computational Molecular Science}~ (WCMS), you will need to use
the~Vancouver reference style, so before exporting click
Export-\textgreater{} Options and select a Vancouver export style.~

\selectlanguage{english}
\FloatBarrier
\section*{References}\sloppy
\phantomsection
\label{csl:1}{Understanding institutions for water allocation and exchange: Insights from dynamic agent-based modeling}. (2019). \textit{Wiley Interdisciplinary Reviews: Water}, \textit{6}(6). \url{https://doi.org/10.1002/wat2.1384}

\end{document}

